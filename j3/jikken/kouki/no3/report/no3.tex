\documentclass{jarticle}

\begin{document}

\section{目的}
H8マイコンのA/D、D/A変換器について理解を深め、それらを用いて割り込みによるデータ入出力プログラムを作成する。

\section{原理}
D/A、A/D変換の原理については前期実験の実験4「H8マイコンのA/D、D/A変換を用いた音声録音再生実験」を参照のこと。

\section{使用機器}
\begin{enumerate}
\item H8マイコン(メーカ:Beyond The River、
\item USBケーブル
\item パーソナルコンピュータ
\end{enumerate}

\section{実験方法}
D/A、A/Dに関わるファイル及び関数定義などは、前期実験の実験4「H8マイコンのA/D、D/A変換を用いた音声録音再生実験」を参照のこと。

実験準備として、「/home/class/j3/jikken/kouki/no3」以下のファイルを各自コピーした上で、以下の課題についてプログラム(rec.c)を作成し動作確認を行う。

\subsection{音声の記録再生1}
H8マイコンボード上の「*」キーを押すと録音モードとなり、「#」キーを押すと再生モードとなるプログラム(ad-da.c)を作成しなさい。スピーカーは録音モード時にマイクとして機能し、再生モード時にはスピーカとして機能するようにしなさい。マイクに向かって音声を入力すると音声データが保存され、その後、スピーカにより再生が可能になる。

\subsection{音声の記録再生2}
H8マイコンボード上の録音モード「*」キーを押すと、上段のLCD表示が「Push * or # key」、下段のLCD表示が「1    」、「12  」、…、「12345」になるようにする。また、再生モード「#」キーを押すと、上段のLCD表示が「Push * or # key」、下段のLCD表示が「1    」、「12  」、…、「12345」、「Push * or # key」になるように作成しなさい。なお、「Now Playing…」、「Now Sampling…」は表示しない。また、1、2、3、4、5の表示は、定義されたTIMEを均等時間になるようにする。

\subsection{音声の録音再生3}
H8マイコンボード上の録音モード「1」キーを押すと、上段のLCD表示が「Push 1 or 2 key」、下段のLCD表示が「Now Sampling…」となるようにする。再生モード「2」キーを押すと、上段のLCD表示が「Push 1 or 2 key」、下段のLCD表示が「Now Playing…」になるように作成しなさい。つまり、前節での録音モード「*」キーを「1」キーに、再生モード「#」キーを「2」に切り替えなさい。なお、駆動時間はTIMEとする。

\subsection{音声の記録再生・逆再生}
H8マイコンボード上の録音モード「*」キーを押すと、上段のLCD表示が「Push * or 5 key」、下段のLCD表示が「Now Sampling…」を表示してTIME時間に録音する。逆再生モード「5」キーを押すと、上段のLCD表示が「Push * or 5 key」、下段のLCD表示が「Now Inverse…」になるようにして、逆再生モードがTIME=12000~0に設定して逆再生するプログラムを作成しなさい。

\subsection{音声の記録再生・統合}
H8マイコンボード上の録音モード「*」キーを押すと、上段のLCD表示が「Push *, # or 5」、下段のLCD表示が「Now Sampling…」を表示してTIME時間に録音する。再生モード「#」キーを押すと、上段のLCD表示が「Push *, # or 5」、下段のLCD表示が「Now Playing…」を表示してTIME時間に再生する。逆再生モード「5」キーを押すと、上段のLCD表示が「Push *, # or 5」、下段のLCD表示が「Now Inverse…」を表示してTIME時間に逆再生するプログラムを作成しなさい。

\section{実験結果}

\subsection{音声の記録再生1}
要件を満たす動作をするプログラムを作成した。ソースファイルをリスト1に、Makefileをリスト2に示す。

\subsection{音声の記録再生2}
要件を満たす動作をするプログラムを作成した。ソースファイルをリスト3に、Makefileはリスト2中の「ad-da」を「no2」に置換したものである。

\subsection{音声の録音再生3}
要件を満たす動作をするプログラムを作成した。ソースファイルをリスト4に、Makefileはリスト2中の「ad-da」を「no3」に置換したものである。

\subsection{音声の記録再生・逆再生}
要件を満たす動作をするプログラムを作成した。ソースファイルをリスト5に、Makefileはリスト2中の「ad-da」を「no4」に置換したものである。

\subsection{音声の記録再生・統合}
要件を満たす動作をするプログラムを作成した。ソースファイルをリスト6に、Makefileはリスト2中の「ad-da」を「no5」に置換したものである。



\end{document}

