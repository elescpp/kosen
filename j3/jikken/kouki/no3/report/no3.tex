\documentclass{jarticle}

\usepackage[top=30truemm,bottom=30truemm,left=25truemm,right=25truemm]{geometry}

\usepackage{graphicx}
\usepackage{listing,jlisting}
\usepackage{ascmac}
\usepackage{here}
\usepackage{txfonts}
\usepackage{listings, jlisting}
\renewcommand{\lstlistingname}{リスト}
\lstset{language=c,
	basicstyle=\ttfamily\scriptsize,
	commentstyle=\textit,
	classoffset=1,
	keywordstyle=\bfseries,
	frame=tRBl,
	framesep=5pt,
	showstringspaces=false,
	numbers=left,
	stepnumber=1,
	numberstyle=\tiny,
	tabsize=2
}
\begin{document}

\section{目的}
H8マイコンのA/D、D/A変換器について理解を深め、それらを用いて割り込みによるデータ入出力プログラムを作成する。

\section{原理}
D/A、A/D変換の原理については前期実験の実験4「H8マイコンのA/D、D/A変換を用いた音声録音再生実験」を参照のこと。

\section{使用機器}
\begin{enumerate}
	\item H8マイコン(メーカ:Beyond The River, 型番:H8-3052)
	\item USBケーブル
	\item パーソナルコンピュータ(メーカ:Dell,型番:Intel Celeron CPU G1820@2.70GHz)
\end{enumerate}

\section{実験方法}
D/A、A/Dに関わるファイル及び関数定義などは、前期実験の実験4「H8マイコンのA/D、D/A変換を用いた音声録音再生実験」を参照のこと。

実験準備として、「/home/class/j3/jikken/kouki/no3」以下のファイルを各自コピーした上で、以下の課題についてプログラム(rec.c)を作成し動作確認を行う。

\subsection{音声の記録再生1}
H8マイコンボード上の「*」キーを押すと録音モードとなり、「\#」キーを押すと再生モードとなるプログラム(ad-da.c)を作成しなさい。スピーカーは録音モード時にマイクとして機能し、再生モード時にはスピーカとして機能するようにしなさい。マイクに向かって音声を入力すると音声データが保存され、その後、スピーカにより再生が可能になる。

\subsection{音声の記録再生2}
H8マイコンボード上の録音モード「*」キーを押すと、上段のLCD表示が「Push * or \# key」、下段のLCD表示が「1    」、「12  」、…、「12345」になるようにする。また、再生モード「\#」キーを押すと、上段のLCD表示が「Push * or \# key」、下段のLCD表示が「1    」、「12  」、…、「12345」、「Push * or \# key」になるように作成しなさい。なお、「Now Playing…」、「Now Sampling…」は表示しない。また、1、2、3、4、5の表示は、定義されたTIMEを均等時間になるようにする。

\subsection{音声の録音再生3}
H8マイコンボード上の録音モード「1」キーを押すと、上段のLCD表示が「Push 1 or 2 key」、下段のLCD表示が「Now Sampling…」となるようにする。再生モード「2」キーを押すと、上段のLCD表示が「Push 1 or 2 key」、下段のLCD表示が「Now Playing…」になるように作成しなさい。つまり、前節での録音モード「*」キーを「1」キーに、再生モード「\#」キーを「2」に切り替えなさい。なお、駆動時間はTIMEとする。

\subsection{音声の記録再生・逆再生}
H8マイコンボード上の録音モード「*」キーを押すと、上段のLCD表示が「Push * or 5 key」、下段のLCD表示が「Now Sampling…」を表示してTIME時間に録音する。逆再生モード「5」キーを押すと、上段のLCD表示が「Push * or 5 key」、下段のLCD表示が「Now Inverse…」になるようにして、逆再生モードがTIME=12000~0に設定して逆再生するプログラムを作成しなさい。

\subsection{音声の記録再生・統合}
H8マイコンボード上の録音モード「*」キーを押すと、上段のLCD表示が「Push *, \# or 5」、下段のLCD表示が「Now Sampling…」を表示してTIME時間に録音する。再生モード「\#」キーを押すと、上段のLCD表示が「Push *, \# or 5」、下段のLCD表示が「Now Playing…」を表示してTIME時間に再生する。逆再生モード「5」キーを押すと、上段のLCD表示が「Push *, \# or 5」、下段のLCD表示が「Now Inverse…」を表示してTIME時間に逆再生するプログラムを作成しなさい。

\section{実験結果}

\subsection{音声の記録再生1}
要件を満たす動作をするプログラムを作成した。ソースファイルをリスト1に、Makefileをリスト2に示す。

\subsection{音声の記録再生2}
要件を満たす動作をするプログラムを作成した。ソースファイルをリスト3に示す。Makefileはリスト2中の「ad-da」を「no2」に置換したものである。

\subsection{音声の録音再生3}
要件を満たす動作をするプログラムを作成した。ソースファイルをリスト4に示す。Makefileはリスト2中の「ad-da」を「no3」に置換したものである。

\subsection{音声の記録再生・逆再生}
要件を満たす動作をするプログラムを作成した。ソースファイルをリスト5に示す。Makefileはリスト2中の「ad-da」を「no4」に置換したものである。

\subsection{音声の記録再生・統合}
要件を満たす動作をするプログラムを作成した。ソースファイルをリスト6に示す。Makefileはリスト2中の「ad-da」を「no5」に置換したものである。


\section{検討課題}
\subsubsection*{・検討課題1}
割り込みのサンプリング間隔を変更すると、録音・再生する音の情報にどのような影響を与えるか、考察せよ。音情報の変化と再現性などをふまえて考察すること。


サンプリング間隔を短くすると、音質は上がるがある容量で録音できる時間が短くなる。長くした場合は音質は下がるが、録音できる時間は長くなる。音源の周波数が高いほど、短いサンプリング間隔が要求される。

\subsubsection*{・検討課題2}
A/Dのサンプリング間隔とD/Aの再生間隔を一致させなかった場合、出力の音情報はどういう変化をするのか、A/Dに対してD/Aを短くする/長くするの2つの観点から検討しなさい。


短くした場合は周波数が高くなるために音が高く聞こえる。長くしたその逆で低く聞こえる。


\lstinputlisting[caption=課題1のソース]{../no1.c}
\lstinputlisting[caption=Makefile]{../make-no1}
\lstinputlisting[caption=課題2のソース]{../no2.c}
\lstinputlisting[caption=課題3のソース]{../no3.c}
\lstinputlisting[caption=課題4のソース]{../no4.c}
\lstinputlisting[caption=課題5のソース]{../no5.c}


\end{document}

